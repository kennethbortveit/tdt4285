\section{Status}
For this section, we show a table containing each part of  SLA and whether it has been accomplished: \\
\begin{tabular}{ | p{5cm} | p{5cm} | p{5cm}| }
\hline
\textbf{SLA section} & \textbf{Description} & \textbf{Accomplished} \\ \hline
Operation Period & Services available from 11th of April to 20th of April 2012 & Yes. There was not any day in which the server was not available. Furthermore, we have not received problems in our system. \\ \hline
Availability & All subsystems are available and working as agreed. & Yes. There were not downtime periods and it has been uptime a total of 100\% of operation period. \\ \hline
Reliability & Monitor the system in order to make sure that interruptions and discovered. & Yes. Database and log errors have not recorded any interruption in the server. \\ \hline
Customer support & The service provider shall provide statistics of request handling at the end of the operation period & Yes. They are showed in the prior section. Problems have not been reported so we did not have to respond to any customer problem. \\ \hline
Change management & The service provider shall follow good change management practice. & We did not have to follow any change management. \\ \hline
Documentation & Documentation regarding the services should be available in the Wiki. & Yes. It is actually in wiki Main Page concerning on “Personal Websites” \\ \hline
Security & The service provider shall make sure that the system is secured and adheres to IDI and NTNU IT regulations. & Yes. Server is configured in order to deny writing requests to addresses which are not the localhost. Also permissions are configured properly in NGINX in order to control this. \\ \hline
\end{tabular}
\pagebreak