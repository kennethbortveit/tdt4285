\section{Self Assessment}
\subsection{Teamwork}
We are a team that probably has very different ways of working and that made it somewhat difficult to collaborate. Although this is a valuable experience, our delivery or our work takes the fall for it. Since this is a school project it does not have any negative effects for other people than the students taking this course and shows us that it is an important skill to adapt to new ways of working together. We worked best when we worked together at the same location. We could easily make sure that everyone was up to date and coordinate assignments to each team member. Sometimes it was not that easy to schedule a meet that suited for all team members and we had to work separate. This would usually work for short periods, but as time went by it got harder to coordinate and put together our different assignments. 
\subsection{Communication}
Our main communication channel was email and an irc chat for inter communication. As for email within the group it is hard to predict if this is an appropriate method to be used. From previous experience email has worked great, but after a while seems that everyone is expecting that you read every email that you get. This usually ends tons of reading material every morning, midday and nights. From this experience it would be better to focus on team meetings as the main communication channel and to ensure that everyone is up to date and understood. This requires more time, but builds team spirit and eventually improves productivity. The other communications methods should not be excluded, but rather be used in addition to the main communication channel. As for the inter communication the OS-group did a very good job setting up the irc channel. There was very little delay when we needed answers regarding the operating system and the servers. At times it could be hard to explain in plain text compared to a meeting, but they made it possible to arrange meetings so it was not really a big problem. Between other groups we primarily used email. It was a little hard to trust that everyone had read and understood what we had was trying to express, but we didn’t experience that this was an issue. 
\subsection{Kenneth}
I would say that our team had some language barriers at first. Not that it made some of the work impossible, but rather a bit more difficult to communicate. In the operation period I was not present due to illness. I think that it wasn’t arranged for that to happen and it would be wise to make this kind of arrangements in the future. As for the workload of each team member I think it was very uneven at some points and it’s difficult to make changes when we have very different expectations of each other. This could probably been avoided if we chose a team leader when the workload started to get uneven. He/she would have the responsibility of scheduling, holding team members responsible and coordinate the project next to working as a regular team member.
\subsection{Kent}
I think that 5 people on an assignment like this, is one or two too many. The language barrier is also a problem when people you work with don’t understand or speak english that well. It can lead to a lot of misunderstandings, or that some messages don’t sink in at all. Another problem is when people plan to be absent for a longer period of time, but doesn’t do the tasks expected of them before they leave. This creates tension between group members because the workload gets uneven. As Kenneth mentioned earlier, a group leader should have been elected.
\subsection{Kristy}
The part in which I have been more comfortable is the first one because of my previous skills. 
I had never worked with MediaWiki and Nginx so I have to look for documentation about it to develop the task properly. As Ngnix is relatively new there are not much information and therefore it was more difficult to do this part. 
English has not represented a barrier for me although I would have like to get the report in english instead of norwegian to understand it better.
I think it would have been better if we would have assigned the tasks at the beginning of the project.
\subsection{Jose}
This project has been useful to me as I have learnt how to monitor, manage and track a server and also to understand the difficulties concerning of working in a large group. For example, when my work mates configured MediaWiki and it worked successfully, I broke it. This caused that I spent so much time in reconfigure it (I had to read documentation and do the same work which had already been done by other people of the team). It could have been avoided if we should have performed a good backup plan.  
\pagebreak