\subsection{NGINX}
NGINX is an HTTP and reverse proxy server, as well as a mail proxy server, written by Igor Sysoev. For a long time, it has been running on many heavily loaded Russian sites including Yandex, Mail.Ru, VKontakte, and Rambler. According to Netcraft nginx served or proxied 9.82\% busiest sites in March 2012.\cite{nginx}
\begin{description}
\item{\textbf{Version: }}1.0.14
\end{description}
\subsection{MediaWiki}
MediaWiki is a free software open source wiki package written in PHP, originally for use on Wikipedia. It is now used by several other projects of the non-profit Wikimedia Foundation and by many other wikis, including this website, the home of MediaWiki.\cite{mediawiki}
\begin{description}
\item{\textbf{Version: }}1.18.1
\end{description}
\subsection{MySQL}
MySQL er et SQL-basert databaseadministrasjonssystem som er lisensiert under GPL. Denne databasetjeneren er veldig mye brukt, og er en vesentlig del av LAMP-systemer, hvor M-en står for nettopp MySQL.\cite{mysql}
\begin{description}
\item{\textbf{Version: }}5.5.2
\end{description}
\subsection{PHP}
PHP is a widely-used general-purpose scripting language that is especially suited for Web development and can be embedded into HTML.\cite{php}
\begin{description}
\item{\textbf{Version: }}5.3.10-1~dotdeb.1
\end{description}
\subsection{How the system is tied together}
First off the operating system needs to be installed, which in this case is Debian Squeeze. When this was installed by the OS-group and when we had obtained our usernames with shell and root access, the installation of the services could begin. The first packages to be installed was MySQL, NGINX and PHP. The MySQL installation were supposed to be done by the Gaming-group, but since we were dependent on MySQL to get MediaWiki up and running, we chose to install it and let the Gaming-group configure it later on. All these packages were easy to install with aptitude.

The MediaWiki was dependent on MySQL and PHP to run since it needs somewhere to store data and because it is written in PHP. Therefore MediaWiki was the last thing to install. To install it you download it to the desired folder, extract and install it from the browser.

At last we had to fix all the configuration files and create user spaces for personal websites. All of this is explained how to do in the installation guide in chapter 3.

